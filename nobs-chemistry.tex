\documentclass[letterpaper, 12pt]{article}
\usepackage[margin=1in]{geometry}

\usepackage{amsfonts}
\usepackage{amsmath}
\usepackage{soul}
\usepackage{cancel}
\usepackage{hyperref}
\usepackage{physics}
\usepackage[version=4]{mhchem} 
\hypersetup{
    colorlinks,
    citecolor=black,
    filecolor=black,
    linkcolor=black,
    urlcolor=black
}

\usepackage{DejaVuSans}
%% Another possibility is
%% \usepackage{dejavu}
%% which loads the DejaVu Serif and DejaVu Sans Mono fonts as well
\renewcommand*\familydefault{\sfdefault} %% Only if the base font of the document is to be sans serif
\usepackage[T1]{fontenc}

\begin{document}
\title{NoBS Chemistry}
\author{Jeffrey Wang, et. al}
\date{November 22, 2016-- November 23, 2016\\version 2016.11.23.15:20}
\maketitle


\setcounter{secnumdepth}{1}
\setcounter{section}{0}

\pagenumbering{roman}

\tableofcontents
\clearpage

\section*{Author's Notes}
	\subsection{NoBS}
	NoBS strives for succinct guides that use simple, smaller, relatable concepts to develop a full understanding of overarching concepts.
	\subsection{Dedication}
	Jeffrey Wang: "To all those that helped me in life: this is for you."
	\subsection{Sources}
	This guide borrows certain material from the following sources, which are indicated below and throughout the paper will be referenced by parentheses and their names:
	\begin{itemize}
		\item (Schwartz) Martin Schwartz, University of North Texas
		\item (Silberberg) Martin S. Silberberg,  \textbf{Principles of General Chemistry, 3rd. Ed.}
	\end{itemize}
\clearpage

\pagenumbering{arabic}
	
\clearpage

\part{Atoms, molecules, and ions}
Review your first semester of eighth grade science class.

\clearpage

\part{Stoichiometry}
Woo mix stuff together in proportions to form new things in proportions

Basically chemical equations

\clearpage

\part{Chemical reactions}
There are three kinds: ionic, acid-base, and redox.

\clearpage

\part{Gases}

$$PV = nRT$$

\clearpage

\part{Introductory thermochemistry}

\section{Hess's Law}
\st{Hess's Law? More like Hiss's Law.}

Enthalpy is a state function. This means $\Delta H$ is independent of path.

Take $3 + 2$. The answer is $5$, right? Well, if I started out with 3, added 7, subtracted 6, and added 1 ($3 + 7 - 6 + 1$), the answer is also $5$. While the middle paths are not the same, the beginning and the end result are the same. This is what a state function is.

Hess's Law takes advantage of the fact that enthalpy is a state function to calculate the enthalpies of reactions that we don't know. No matter how steps there are in a series of chemical reactions, the total enthalpy change is the sum of all the changes.

There are a few basic ground rules that come from enthalpy being a state function:
\begin{enumerate}
	\item If reactants and products are reversed in a chemical equation, flip the sign of $\Delta H$.
	\begin{itemize}
		\item \ce{CO2(g) + 2H2O(l) -> CH4(g) + 2O2(g)} $\Delta H_{1} = +890 kJ$
		\item \ce{CH4(g) + 2O2(g) -> CO2(g) + 2H2O(l)} $\Delta H_{2} = -890 kJ$
	\end{itemize}
	\item If the amounts of reactants and products are changed by a certain ratio, then $\Delta H$ is changed in the same ratio.
	\begin{itemize}
		\item \ce{CO2(g) + 2H2O(l) -> CH4(g) + 2O2(g)} $\Delta H_{1} = +890 kJ$
		\item \ce{2CO2(g) + 4H2O(l) -> 2CH4(g) + 4O2(g)} $\Delta H_{2} = +1780 kJ$
	\end{itemize}
\end{enumerate}

We can combine these two rules. If the $\Delta H$ of \ce{H2(g) + Cl2(g) -> 2HCl(g)} is $-185 kJ$, then what is the $\Delta H$ of \ce{4HCl(g) -> 2H2(g) + 2Cl2(g)}?

We can identify two differences between these two reactions. First, it looks like the reaction was flipped. Second, it looks like the coefficients were multiplied by two. To account for these differences, we will (1) flip the sign of $\Delta H$ and (2) multiply $\Delta H$ by 2, which is the ratio that the coefficients were multiplied by. So the final $\Delta H$ is $+370 kJ$.

\clearpage
	
\part{Quantum theory and atomic structure}

\section{Waves}
	\begin{itemize}
		\item A wave is an oscillation that results in a movement of energy. Light has characteristics of both waves and particles.
		\item Two characteristics of waves
		\begin{itemize}
			\item \textbf{wavelength} ($\lambda$, length) - distance between wave peaks
			\item \textbf{frequency} ($\nu$, per time) - number of peaks that pass a point in one second - HEY BTW "$\nu$" IS NOT "v" IT'S CALLED NU kthxbai jk get back to studying
		\end{itemize}
		\item Frequency related inversely to wavelength: $$\nu \propto \frac{1}{\lambda}$$
		\item Product of wavelength, frequency is \textbf{speed of light} (c, length per time)
		\begin{itemize}
			\item Relationship: $\lambda \nu = c = 2.998 \times 10^8 m/s = 3.00 \times 10^8 m/s$
			\item Remember inverse relationships were $xy = k$? The "k" in this case is $c$, or the speed of light.
			\item For any given wavelength, divide c by the wavelength to get the frequency.
			\item Also, for any frequency, divide c by the frequency to get wavelength.
			\item This c is seen in the famous equation $E = mc^2$.
		\end{itemize}
	\end{itemize}
\section{Quantum theory}
	\subsection{Energy}
	\begin{itemize}
		\item Planck discovered energy comes in discrete "packets" called quanta. (Ergo, quantum theory.) So you can't break it apart. Just like how atoms are the smallest subdivision of matter, quanta are the smallest subdivision of energy.
		\item This is how we determine how much energy something truly has.
		\item Planck's constant = number of joules of energy a quantum (packet of energy) has = h = $6.63 \times 10^{-34} J \cdot s$
		\item When electrons move down energy levels, they lose energy. This energy goes into photons, which are emitted from atoms. Photons have wave-like properties. Therefore, we can use that wave stuff to figure out the energy given by photons.
		\item Energy of each photon proportional to light frequency: $E_{photon} \propto \nu$
		\item $E_{photon} = h \nu = \frac{hc}{\lambda}$. (You can substitute $\nu$ for $\frac{c}{\lambda}$ because $\nu = \frac{c}{\lambda}$, per the waves section we just went over.)
		\item Multiply energy of a photon by Avogadro's number and hey look what you get, a mole of energy
	\end{itemize}
	\subsection{Quantum numbers}
	\begin{itemize}
		\item Use quantum numbers to determine what electron is in what orbital
		\item Four quantum numbers
		\item \textbf{$n$} - \textbf{principal quantum number}
		\begin{itemize}
			\item $n = 1, 2, 3, ...$
			\item Determines energy level and size of orbital (Lower energy level vs higher energy level. Higher energy levels means larger orbitals)
		\end{itemize}
		\item \textbf{$l$} - \textbf{azimuthal quantum number}
		\begin{itemize}
			\item $l = 0, 1, 2, ..., n-1$
			\item Determines number and shape of orbitals
			\item Better known as spdf
		\end{itemize}
		\item \textbf{$m_{l}$} - \textbf{magnetic quantum number}
		\begin{itemize}
			\item $m_{l} = -l, ..., -1, 0, 1, ..., l$
			\item Determines orientation of orbitals (Basically, only two electrons per orbital, so you have to make different "orientations" of the same orbital for each two electrons)
		\end{itemize}
		\item \textbf{$m_{s}$} - \textbf{spin quantum number}
		\begin{itemize}
			\item $m_{s} = -\frac{1}{2}, \frac{1}{2}$
			\item Determines "spin" of electron (because you get two electrons to an orbital, but they have to be going different ways, so this is how it's identified as going one way or another)
		\end{itemize}
		\item \textbf{Pauli Exclusion Principle} - no two electrons can share the same set of quantum numbers. (For example, this means they can have the same principal, azimuthal, and magnetic quantum numbers, but the couldn't have the same spin quantum number!)
	\end{itemize}
\clearpage

\setcounter{part}{999}
\setcounter{secnumdepth}{1}
\setcounter{section}{0}
\part{Practice exercises}

\newtheorem{problem}{Problem}[section]

\section{Acid-base equilibria}

\begin{problem}
Amil adds 100mL of 1.75M \ce{HC2H3O2} into his daily juice blend. Assuming he drinks this juice blend thrice a day and his stomach pH before drinking the juice is 2.3, what is the pH of his stomach after all three juices were swallowed? (Amil's stomach can hold 1L of liquid. He has 50mL of stomach acid. Assume he fills up his stomach completely and nothing flows into his duodenum between drinking each juice.)
\end{problem}

\begin{problem}
DK adds 3.50M of \ce{NH4Cl} into his daily juice blend. Assuming he drinks this juice blend once a day and his stomach pH before drinking the juice is 2.1, what is the pH of his stomach after he swallows the juice? (DK drinks 100mL of ammonium chloride per cup, and his stomach can hold 1L of content. He has 50mL of stomach acid. Assume he fills up his stomach completely. Remember stomach acid is \ce{HCl}.)
\end{problem}

\end{document}
